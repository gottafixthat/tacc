\chapter{Chart of Accounts}
\label{Chart of Accounts}
The Chart of Accounts determines how sales and other financial reports
are generated from within TACC.  Each account in the Chart of Accounts
must be of the specific type laid out in the Account Types (see page
~\pageref{QuickStartAccountTypes}) and have an Account Number associated
with it.

Account Numbers should be grouped together within a numerical range for
a specific type of account.  For example, all accounts that are
\emph{Assets} are normally found in the range 1000-1999, \emph{Income}
accounts are normally 3000-3999, etc.  You can have duplicate account
numbers within TACC, but they will be treated as individual items.
{\bf It his highly recommended that you make all account numbers unique}.

Account Numbers are free-form within TACC and do not actually have to
specify a number.  TACC employs a "mask" that determines how the account
numbers must be entered and will be displayed.  If you wanted to
extended TACC to use an 8 digit account number, with the first four
digits being the standard accounting account number, the next two being
the company, and the final two being the location (which is the default
setting), you would use a mask to do this.

The default mask is {\tt D999-99-99}, where "D" is a digit between 1 and 9,
and "9" is any digit between 0 and 9.  The dashes are automatically used
and inserted.  For information on changing the account mask and the
possible values, please see the {\it TACC Administrators Guide}.

